\documentclass{article}
\usepackage[utf8]{inputenc}

\title{TKK DN2}
\author{Janez Justin, 63180016}
\usepackage{amssymb}
\usepackage{natbib}
\usepackage{graphicx}
\usepackage{amsmath}
\usepackage{seqsplit}
\usepackage{amssymb}
\begin{document}

\maketitle

\section{Geffejev generator}
\subsection{Podatki}
Niz bitov predstavim kot seznam, ki vsebuje 0 in 1. Začetni ključ predstavim, kot seznam dolžine $m$, kjer je register predstavljen s polinomom stopnje $m$. Polinome sem "vgradil" kar v funkcije, ki predstavljajo posamezen register.
\subsection{Registra 1 in 3}
Ker se $\frac{3}{4}$ izhodov registrov ponovi v generatorju, upam na to, da se ob "najuspešnejšem" ključu generira zaporednje, ki ga prištejem kriptogramu, kar mi izpljune besedilo z najvecjim številom legalnih znakov (torej najmanj peterk bitov, ki predstavljajo števila med 26 in 32). Tako pregledam vse ključe in izbrem tistega, pri katerem dobim najmanj ilegalnih znakov.
\subsection{Regsiter 2}
Pretečem vse ključe in izberem tistega, s katerim lahko odkriptiram besedilo brez ilegalnih znakov.
\subsection{Opombe}
\begin{itemize}
    \item Znalo bi se zgoditi, da bi moral za registra 1 in 3 shraniti nekaj najboljših ključev in pregledati njihove kombinacije.
    \item Pri registru 2 je možno, da bi za kakšen drug kriptogram dobili več možnih besedil. Ta problem bi bil rešljiv s kriptoanalizo teh besedil.
\end{itemize}
\subsection{Rezultat}
Priloženi kriptogram se prevede v:\\
   \textbf{\seqsplit{CRYPTOGRAPHYPRIORTOTHEMODERNAGEWASEFFECTIVELYSYNONYMOUSWITHENCRYPTIONTHECONVERSIONOFINFORMATIONFROMAREADABLESTATETOAPPARENTNONSENSETHEORIGINATOROFANENCRYPTEDMESSAGEALICESHAREDTHEDECODINGTECHNIQUENEEDEDTORECOVERTHEORIGINALINFORMATIONONLYWITHINTENDEDRECIPIENTSBOBTHEREBYPRECLUDINGUNWANTEDPERSONSEVEFROMDOINGTHESAMETHECRYPTOGRAPHYLITERATUREOFTENUSESALICEAFORTHESENDERBOBBFORTHEINTENDEDRECIPIENTANDEVEEAVESDROPPERFORTHEADVERSARYSINCETHEDEVELOPMENTOFROTORCIPHERMACHINESINWORLDWARIANDTHEADVENTOFCOMPUTERSINWORLDWARIITHEMETHODSUSEDTOCARRYOUTCRYPTOLOGYHAVEBECOMEINCREASINGLYCOMPLEXANDITSAPPLICATIONMOREWIDESPREAD}}\\
   s ključi:\\\textbf{1: 01110\\
   2: 1101001\\
   3: 11110011010}
\end{document}
